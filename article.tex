\documentclass[a4paper, twocolumn]{article}


% you can switch between these two (and more) styles by commenting one out (use percentage)
\usepackage[backend=biber]{biblatex}
%\usepackage[backend=biber, style=authoryear-icomp]{biblatex}
\addbibresource{./refs.bib}

\usepackage{graphicx}

\usepackage{listings}
\usepackage{color}
\definecolor{lightgray}{gray}{0.9}

% code listing: https://tex.stackexchange.com/questions/19004/how-to-format-an-inline-source-code
\lstset{
    showstringspaces=false,
    basicstyle=\ttfamily,
    keywordstyle={blue},
    commentstyle=\color[gray]{0.6}
    stringstyle=\color[RGB]{255, 150, 75}
}
\newcommand{\inlinecode}[2]{\colorbox{lightgray}{\lstinline[language=#1]$#2$}}

\author{Joachim Richter et al.}
\title{Using \LaTeX}



\begin{document}

\twocolumn[
    \begin{@twocolumnfalse}
        \maketitle
        \begin{abstract}
            Is it possible to predict the outcome of a car's miles per gallon based on various factors such as the number of cylinders, displacement, horsepower, weight, acceleration, model year, origin, and car name? 
            Is it possible to predict a car's make based on the same factors? 
            This paper explores these questions using linear regression and logistic regression.
        \end{abstract}
    \end{@twocolumnfalse}
    \vspace{1cm}
]


\section{Introduction\label{sec:Introduction}}

We identified missing datapoints in horsepower replaced with a '?'. 
We decided, that since the dataset was so small, that we would manually infer the horsepower based on looking at other cars, with similar values, and deduct the missing horsepower. 

\section{Research Question\label{sec:Research Question}}

The research question must be phrased as a question, ending with "?".
It is often a good idea to structure your research question into one overall question with a number of sub-questions, and you can number them i.e. Q1, Q2 etc. to make it easy to refer to them.
Alternatively, you can phrase a hypothesis instead of a research question.

\section{Methods\label{sec:Methods}}

The methods section is for documenting the methods you apply to conduct your research.
It is about the methods and how they will be applied, but not about their actual application.
Ideally, you should write your method section before you start the actual research.
The methods section must be well referenced.

\section{Analysis\label{sec:Analysis}}

In the analysis section you apply the methods you described in the method section.
It is specific to your particular paper and your particular research question.
Make sure to make cross references back to your method section.

\section{Findings\label{sec:Findings}}

Here you present your findings, that is what came out of your analysis.
Make sure to cross reference back to your analysis section.

\section{Conclusion\label{sec:Conclusion}}

In the conclusion you answer your research question based on your findings.
Make sure to make cross references to your research question, analysis, and findings sections.

\printbibliography

\end{document}
