\documentclass[a4paper, twocolumn]{article}


% you can switch between these two (and more) styles by commenting one out (use percentage)
\usepackage[backend=biber]{biblatex}
%\usepackage[backend=biber, style=authoryear-icomp]{biblatex}
\addbibresource{./refs.bib}

\usepackage{graphicx}

\usepackage{listings}
\usepackage{color}
\definecolor{lightgray}{gray}{0.9}

% code listing: https://tex.stackexchange.com/questions/19004/how-to-format-an-inline-source-code
\lstset{
    showstringspaces=false,
    basicstyle=\ttfamily,
    keywordstyle={blue},
    commentstyle=\color[gray]{0.6}
    stringstyle=\color[RGB]{255, 150, 75}
}
\newcommand{\inlinecode}[2]{\colorbox{light gray}{\lstinline[language=#1]$#2$}}

\author{Jeffrey Roed, Joachim Richter, and Rasmus Kibshede}
\title{Predicting Car Mileage and Make using Linear and Logistic Regression}



\begin{document}


\twocolumn[
    \begin{@twocolumnfalse}
        \maketitle
        \begin{abstract}
            
        \end{abstract}
    \end{@twocolumnfalse}
    \vspace{1cm}
]


\section{Introduction\label{sec:Introduction}}
group name: Silicon

% The introduction must put your paper into a context.
% The reader should be able to understand your paper based on the background information provided in the introduction.

\section{Research Question\label{sec:Research Question}}
Q1: Is it possible to predict the outcome of a car's miles per gallon based on various factors such as the number of cylinders, displacement, horsepower, weight, acceleration, model year, origin, and car name?\\
Q2: Is it possible to predict a car's make based on the same factors? 

% The research question must be phrased as a question, ending with "?".
% It is often a good idea to structure your research question into one overall question with a number of sub-questions, and you can number them i.e. Q1, Q2 etc. to make it easy to refer to them.
% Alternatively, you can phrase a hypothesis instead of a research question.

\section{Methods\label{sec:Methods}}

pair plot

In this study, linear regression and logistic regression were applied to predict the outcome of a car's miles per gallon and make, respectively. Linear regression utilized factors such as the number of cylinders, displacement, horsepower, weight, acceleration, model year, origin, and car name to predict miles per gallon. Logistic regression used the same factors to predict the car's make.

The dataset, provided by Henrik Strøm, an AI teacher at KEA, included details on car performance and make. Initial examination revealed faulty and deficient data. Using well know data cleaning techniques such as Preprocessing \textcite{hellerstein2013quantitative}

Following the data completion, feature engineering was conducted. This process transformed categorical variables, such as make and model, into numerical values through the use of one-hot encoding, enabling their inclusion in the regression analysis.

Different machine learning models were considered, and their performances were gauged using metrics such as mean squared error, mean absolute error, and root mean squared error.



% The methods section is for documenting the methods you apply to conduct your research.
% It is about the methods and how they will be applied, but not about their actual application.
% Ideally, you should write your method section before you start the actual research.
% The methods section must be well referenced.

\section{Analysis\label{sec:Analysis}}

% In the analysis section you apply the methods you described in the method section.
% It is specific to your particular paper and your particular research question.
% Make sure to make cross references back to your method section.

\section{Findings\label{sec:Findings}}

% Here you present your findings, that is what came out of your analysis.
% Make sure to cross reference back to your analysis section.

\section{Conclusion\label{sec:Conclusion}}

% In the conclusion you answer your research question based on your findings.
% Make sure to make cross references to your research question, analysis, and findings sections.

\end{document}
